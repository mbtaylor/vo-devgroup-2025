\documentclass[20pt,landscape]{foils}
\usepackage{mbtslides}
\usepackage[english]{babel}
\usepackage{graphicx}
\usepackage{marvosym}
\usepackage{amssymb}% provides \gtrsim
\usepackage{ellipse}
\usepackage{fancyvrb}
\usepackage{ulem}% provides \sout
\usepackage[cm]{sfmath}

\newif\ifrubric
% \rubrictrue

%This is supposed to make underscores cut'n'pastable
%Not sure what the lmodern package does or whether it's needed for this.
%But it's not working (though apparently it did in nam2023)
%\usepackage[T1]{fontenc}
%\usepackage{lmodern}
%\usepackage{textcomp}
%\DeclareTextSymbolDefault{\textunderscore}{T1}

\setlength{\unitlength}{1cm}


\newcommand{\bhref}[2]{\href{#1}{{\color{blue}#2}}}
\newcommand{\burl}[1]{{\color{blue}\url{#1}}}

% Fixes problem with includegraphics images screwing up colours on their
% page in the output PDF.  I have no idea *how* it fixes it mind.
\pdfpageattr {/Group << /S /Transparency /I true /CS /DeviceRGB>>}
  
% Oh no, this is required on my Ubuntu 22.04 (though not 20.04) installation
% to make the \ellipse commands work.
\makeatletter
\newdimen\@tempdimd
\makeatother

\begin{document}
\sf
\newcommand{\bigword}[1]{
  \vspace*{7cm}
  \begin{center}
    \color{darkblue}
    \scalebox{3}{
      \Huge\bf #1
    }
  \end{center}
  \addtocounter{page}{-1}
}

\ifrubric

\rightfooter{}
\MyLogo{}

\vspace*{3cm}
\hspace*{5cm}
\begin{minipage}{30cm}
\LARGE
\begin{enumerate}
  \item Wins
  \item Issues
  \item Intro to the Virtual Observatory
  \item AOB
\end{enumerate}
\end{minipage}

\newpage
\bigword{Wins?}
\newpage
\bigword{Issues?}
\newpage

\fi

\MyLogo{\color{grey}
        Mark Taylor, Introduction to the Virtual Observatory,
        Bristol Astro Dev Group,
        12 December 2025}
\rightfooter{\quad{\color{grey}\thepage/\pageref*{lastPage}}}
\setcounter{page}{1}

\begin{picture}(30,0)
  \put(25,-13){\includegraphics[width=12cm]{ivoa_logoc.jpg}}
\end{picture}

\vspace*{-0.5cm}
\begin{center}
{\color{darkblue}
\framebox{\Huge\bf
  \begin{minipage}{0cm}
  \begin{tabbing}
  Introduction to the \\
  Virtual Observatory
  \end{tabbing}
  \end{minipage}
}}
\\[2.0cm]
{\Large
  Mark Taylor
}
\\[2.0cm]
{\large\color{grey}
  Bristol Astro Dev Group
  \\[2ex]
  12 December 2025
}
\end{center}

\vspace*{1.5cm}
\begin{center}
  \tiny
  \color{brown}
  \input gitid
\end{center}

\newpage

\vspace*{6cm}
\begin{center}
\begin{minipage}{0cm}
\Huge
\begin{tabbing}
What is the Virtual Observatory? \\[2ex]
And why should I care?
\end{tabbing}
\end{minipage}
\end{center}

\newpage

\begin{list0}
  \item {\color{darkred}\sl ``All astro archives in your computer''}
  \item What the VO is NOT:
  \begin{list2big}
    \item A massive computer holding all archived datasets
  \end{list2big}
  \item What the VO is:
  \begin{list2big}
    \item A set of {\color{darkred} interoperable protocols}
          that clients and services can use to talk to each other
          {\color{darkred} in a uniform way}
  \end{list2big}
\end{list0}

\begin{list0}
  \item What the VO also is not:
  \begin{list2big}
     \item Astroquery
  \end{list2big}
\end{list0}

\newcommand{\archSlide}[1]{
\slide{Virtual Observatory Architecture Diagram v2.1}

\begin{center}
  \includegraphics[height=15cm]{#1}

  \burl{https://www.ivoa.net/documents/IVOAArchitecture/}
\end{center}
}
\archSlide{archdiag0.png}
\archSlide{archdiag1.png}
\addtocounter{page}{-1}
\archSlide{archdiag2.png}
\addtocounter{page}{-1}

\slide{Virtual Observatory Protocols}

\begin{picture}(26,0)
  \put(25,-7){\includegraphics[height=8cm]{archdiag2.png}}
  \put(26,-17){{\color{darkred}$^{\ast}$ YMMV}}
\end{picture}
\vspace*{-1.5cm}

\begin{list0}
\vspace*{-0.2cm}
  \item Most important{\color{darkred}$^{\ast}$} user-facing VO standards:
  \begin{list2}
\vspace*{-0.2cm}
    \item Directory service:
    \begin{list3}
      \item {\bf Registry} --- Locate data archives/services
    \end{list3}
\vspace*{-0.2cm}
    \item Positional data access services (by RA/Dec + radius):
    \begin{list3}
      \item {\bf Cone Search} --- Rows from catalogue
      \item {\bf Simple Image Access (SIA)} --- Images from archive
      \item {\bf Simple Spectral Access (SSA)} --- Spectra from archive
      \item {\bf Hierarchical Progressive Survey (HiPS)} --- All-sky imagery
    \end{list3}
\vspace*{-0.2cm}
    \item General data access service:
    \begin{list3}
      \item {\bf Table Access Protocol/Astro Data Query Language (TAP/ADQL)}
            --- SQL-like query of database
      \item {\bf ObsCore}
    \end{list3}
\vspace*{-0.2cm}
    \item Application communications:
    \begin{list3}
      \item {\bf Simple Application Messaging Protocol (SAMP)} ---
            exchange data/control between local applications
    \end{list3}
\vspace*{-0.2cm}
    \item File format:
    \begin{list3}
      \item {\bf VOTable} -- exhanges table data with rich metadata
      \item {\bf Multi-Order Coverage map (MOC)}
            --- compact description of sky coverage
    \end{list3}
\vspace*{-0.2cm}
    \item Semantics:
    \begin{list3}
      \item {\bf Uniform Content Descriptor (UCD)}
            --- describes columns in a table
      \item {\bf VOUnits} -- standard units notation
    \end{list3}
  \end{list2}
% \item {\color{grey}Not quite the VO}
% \begin{list2}
%   \item[{\color{grey}$\bullet$}] {\color{grey}VizieR}
%   \item[{\color{grey}$\bullet$}] {\color{grey}CDS Xmatch service}
% \end{list2}
\end{list0}

\slide{Registry}

\begin{picture}(30,0)
  \put(22,-17){\includegraphics[height=18cm]{registry.png}}
\end{picture}
\vspace*{-1.5cm}

\begin{list0}
  \item Database of all VO Services \\
        (also documents, organisations, data collections...)
  \begin{list2}
    \item It's not actually centralised, but you can think of it as if it was
    \item Tools offering you services will query the Registry to find \\
          where they live and what's in them
    \item You can also query it directly using TAP (RegTAP)\\[2ex]
          {\footnotesize\color{brown}\begin{verbatim}
   SELECT ivoid, short_name, res_title, access_url
   FROM rr.resource
   NATURAL JOIN rr.capability
   NATURAL JOIN rr.interface
   WHERE standard_id = 'ivo://ivoa.net/std/conesearch'
     AND (short_name ILIKE '%sdss%' OR 1=ivo_hasword(res_title, 'sdss'))
     AND (short_name ILIKE '%dr16%' OR 1=ivo_hasword(res_title, 'dr16'))
          \end{verbatim}}
  \end{list2}
\end{list0}

\slide{VOTable}

\slide{Simple positional queries}

\begin{picture}(30,0)
  \put(0,-17){\includegraphics[height=8cm]{m31-cone.pdf}}
  \put(9,-17){\includegraphics[height=8cm]{cone-table.png}}
  \put(24,-17){\includegraphics[height=16cm]{cone-window.png}}
\end{picture}
\vspace*{-1.5cm}

\begin{list0}
  \item \textbf{Cone} Search,
        \textbf{S}imple \textbf{I}mage \textbf{A}ccess,
        \textbf{S}imple \textbf{S}pectral \textbf{A}ccess
  \begin{list2}
    \item Give the service a sky position (RA, Dec) and some kind of radius \\
          (details differ a bit between Cone, SIA, SSA)
    \item The service gives you back a (VO)Table with a row for each result
    \begin{description}
      \item[Cone:] list of catalogue entries (e.g.\ sources) near that position
      \item[SIA:]  list of images near that position
                   (one column has a URL for the image file)
      \item[SSA:]  list of spectra near that position
                   (one column has a URL for the spectrum file)
    \end{description}
  \end{list2}
\end{list0}

\slide{HiPS}

\begin{picture}(30,0)
  \put(22,-17){\includegraphics[width=15cm]{aladin-lite.png}}
\end{picture}
\vspace*{-1.5cm}

\begin{list0}
  \item All-sky hierarchical imagery
        (\textbf{Hi}erarchical \textbf{P}rogressive \textbf{S}urvey)
  \begin{list2big}
    \item Like Google Earth, but looking up (usually)
    \item Based on HEALPix tiling (doesn't blow up at the poles)
    \item HiPS datasets exist for
          lots of sky surveys image archives
    \begin{list3}
      \item Also quite a few non-sky image sets too (Mars, the Moon, ...)
    \end{list3}
    \item Usually access it using a dedicated all-sky viewer, \\
          mostly Aladin variants:
    \begin{list3}
      \item \bhref{https://aladin.cds.unistra.fr/}{Aladin}:
            desktop java tool
      \item \bhref{https://aladin.cds.unistra.fr/AladinLite/doc/}{Aladin Lite}:
            web-based (super easy to embed in your own web pages)
      \item \bhref{https://github.com/cds-astro/ipyaladin}{ipyaladin}:
            stick it in a Jupyter notebook
      \item \bhref{https://sky.esa.int/}{ESASky}:
            AladinLite-based interface to ESA-curated sky survey data
    \end{list3}
    \item You can also do cutouts: 
          \bhref{https://alasky.cds.unistra.fr/hips-image-services/hips2fits}
                {HiPS2FITS}
    \item HiPS exist for some spectral cube datasets too
  \end{list2big}
\end{list0}

\slide{TAP/ADQL}

\slide{ObsCore}

\slide{SAMP}

\slide{MOC}

\slide{UCD/VOUnits}




\label{lastPage}
   
\ifrubric

\newcommand{\aobSlide}[1]{
\newpage
\rightfooter{}
\MyLogo{} 
\begin{picture}(30,0)
  #1
\end{picture} 
\bigword{\ \ AOB?}
}
\aobSlide{}


\fi

\end{document}
