\documentclass[20pt,landscape]{foils}
\usepackage{mbtslides}
\usepackage[english]{babel}
\usepackage{graphicx}
\usepackage{marvosym}
\usepackage{amssymb}% provides \gtrsim
\usepackage{ellipse}
\usepackage{fancyvrb}
\usepackage{ulem}% provides \sout
\usepackage[cm]{sfmath}

\newif\ifrubric
% \rubrictrue

%This is supposed to make underscores cut'n'pastable
%Not sure what the lmodern package does or whether it's needed for this.
%But it's not working (though apparently it did in nam2023)
%\usepackage[T1]{fontenc}
%\usepackage{lmodern}
%\usepackage{textcomp}
%\DeclareTextSymbolDefault{\textunderscore}{T1}

\setlength{\unitlength}{1cm}


\newcommand{\bhref}[2]{\href{#1}{{\color{blue}#2}}}
\newcommand{\burl}[1]{{\color{blue}\url{#1}}}

% Fixes problem with includegraphics images screwing up colours on their
% page in the output PDF.  I have no idea *how* it fixes it mind.
\pdfpageattr {/Group << /S /Transparency /I true /CS /DeviceRGB>>}
  
% Oh no, this is required on my Ubuntu 22.04 (though not 20.04) installation
% to make the \ellipse commands work.
\makeatletter
\newdimen\@tempdimd
\makeatother

\begin{document}
\sf
\newcommand{\bigword}[1]{
  \vspace*{7cm}
  \begin{center}
    \color{darkblue}
    \scalebox{3}{
      \Huge\bf #1
    }
  \end{center}
  \addtocounter{page}{-1}
}

\ifrubric

\rightfooter{}
\MyLogo{}

\vspace*{3cm}
\hspace*{5cm}
\begin{minipage}{30cm}
\LARGE
\begin{enumerate}
  \item Wins
  \item Issues
  \item Intro to the Virtual Observatory
  \item AOB
\end{enumerate}
\end{minipage}

\newpage
\bigword{Wins?}
\newpage
\bigword{Issues?}
\newpage

\fi

\MyLogo{\color{grey}
        Mark Taylor, Introduction to the Virtual Observatory,
        Bristol Astro Dev Group,
        12 December 2025}
\rightfooter{\quad{\color{grey}\thepage/\pageref*{lastPage}}}
\setcounter{page}{1}

\begin{picture}(30,0)
  \put(25,-15){\includegraphics[width=12cm]{ivoa_logoc.jpg}}
\end{picture}

\vspace*{1.5cm}
\begin{center}
{\color{darkblue}
\framebox{\Huge\bf
  \begin{minipage}{0cm}
  \begin{tabbing}
  Introduction to the \\
  Virtual Observatory
  \end{tabbing}
  \end{minipage}
}}
\\[2.0cm]
{\Large
  Mark Taylor
}
\\[2.0cm]
{\large\color{grey}
  Bristol Astro Dev Group
  \\[2ex]
  12 December 2025
}
\end{center}

\vspace*{1.5cm}
\begin{center}
  \tiny
  \color{brown}
  \input gitid
\end{center}

\newpage

\vspace*{6cm}
\begin{center}
\begin{minipage}{0cm}
\Huge\color{teal}
\begin{tabbing}
What is the Virtual Observatory? \\[2ex]
And why should I care?
\end{tabbing}
\end{minipage}
\end{center}

\newpage
\vspace*{0.4cm}
\begin{center}
{\Large\bf\color{darkred}
 ``All astronomy archives in your computer''}
\end{center}

\begin{list0}
  \item What the VO is NOT:
  \begin{list2big}
    \item A massive computer holding all archived datasets
  \end{list2big}
  \item What the VO is:
  \begin{list2big}
    \item A set of {\color{darkred} interoperable standards}
          that clients and services can use to talk to each other
          {\color{darkred} in a uniform way}
    \begin{list3}
      \item Standards are developed and maintained by the IVOA
            (\bhref{https://www.ivoa.net/}
                   {International Virtual Observatory Alliance}) \\
            --- in business since $\sim$2002
    \end{list3}
    \item and software that uses those standards:
    \begin{list3}
      \item lots of data services
      \item clients like PyVO, TOPCAT, STILTS, Aladin(-Lite), SPLAT, CASSIS,
            scripts, web pages, ...
    \end{list3}
  \end{list2big}
  \item You're probably already using it
  \begin{list2big}
    \item VO publication is quite mainstream for large data providers now
    \begin{list3}
      \item e.g.\ if you're using Gaia data,
                  it's probably come from a VO service
    \end{list3}
    \item You don't have to know all about the VO, but:
    \begin{list3}
      \item some background may help you to understand what tools are doing
      \item there may be some useful functionality out there
            you don't know about
    \end{list3}
  \end{list2big}
\end{list0}

% \begin{list0}
%   \item What the VO also is not:
%   \begin{list2big}
%      \item Astroquery
%   \end{list2big}
% \end{list0}

\newcommand{\archSlide}[1]{
\slide{IVOA Architecture Diagram v2.1}

\begin{center}
  \includegraphics[height=15cm]{#1}

  \burl{https://www.ivoa.net/documents/IVOAArchitecture/}
\end{center}
}
\archSlide{archdiag0.png}
\archSlide{archdiag1.png}
\addtocounter{page}{-1}
\archSlide{archdiag2.png}
\addtocounter{page}{-1}

\slide{Virtual Observatory Protocols}

\begin{picture}(26,0)
  \put(25,-7){\includegraphics[height=8cm]{archdiag2.png}}
  \put(26,-17){{\color{darkred}$^{\ast}$ YMMV}}
\end{picture}
\vspace*{-1.5cm}

\begin{list0}
\vspace*{-0.2cm}
  \item Most important{\color{darkred}$^{\ast}$} user-facing VO standards:
  \begin{list2}
\vspace*{-0.2cm}
    \item Directory service:
    \begin{list3}
      \item {\bf Registry} --- Locate data archives/services
    \end{list3}
\vspace*{-0.2cm}
    \item Positional data access services (by RA/Dec + radius):
    \begin{list3}
      \item {\bf Cone Search} --- Rows from catalogue
      \item {\bf Simple Image Access (SIA)} --- Images from archive
      \item {\bf Simple Spectral Access (SSA)} --- Spectra from archive
      \item {\bf Hierarchical Progressive Survey (HiPS)} --- All-sky imagery
    \end{list3}
\vspace*{-0.2cm}
    \item General data access service:
    \begin{list3}
      \item {\bf Table Access Protocol/Astro Data Query Language (TAP/ADQL)}
            --- SQL-like query of database
      \item {\bf ObsCore}
    \end{list3}
\vspace*{-0.2cm}
    \item Application communications:
    \begin{list3}
      \item {\bf Simple Application Messaging Protocol (SAMP)} ---
            exchange data/control between local applications
    \end{list3}
\vspace*{-0.2cm}
    \item File format:
    \begin{list3}
      \item {\bf VOTable} -- exhanges table data with rich metadata
      \item {\bf Multi-Order Coverage map (MOC)}
            --- compact description of sky coverage
    \end{list3}
\vspace*{-0.2cm}
    \item Semantics:
    \begin{list3}
      \item {\bf Uniform Content Descriptor (UCD)}
            --- describes columns in a table
      \item {\bf VOUnits} -- standard units notation
    \end{list3}
  \end{list2}
% \item {\color{grey}Not quite the VO}
% \begin{list2}
%   \item[{\color{grey}$\bullet$}] {\color{grey}VizieR}
%   \item[{\color{grey}$\bullet$}] {\color{grey}CDS Xmatch service}
% \end{list2}
\end{list0}

\slide{Registry}

\begin{picture}(30,0)
  \put(22,-17){\includegraphics[height=18cm]{registry.png}}
\end{picture}
\vspace*{-1.5cm}

\begin{list0}
  \item Database of all VO Services \\
        (also documents, organisations, data collections...)
  \begin{list2}
    \item It's not actually centralised, but you can think of it as if it was
    \item Tools offering you services will query the Registry to find \\
          where they live and what's in them
    \item You can also query it directly using TAP (RegTAP)\\[2ex]
          {\footnotesize\color{brown}\begin{verbatim}
   SELECT ivoid, short_name, res_title, access_url
   FROM rr.resource
   NATURAL JOIN rr.capability
   NATURAL JOIN rr.interface
   WHERE standard_id = 'ivo://ivoa.net/std/conesearch'
     AND (short_name ILIKE '%sdss%' OR 1=ivo_hasword(res_title, 'sdss'))
     AND (short_name ILIKE '%dr16%' OR 1=ivo_hasword(res_title, 'dr16'))
          \end{verbatim}}
    \item Or browse it using \bhref{https://dc.g-vo.org/wirr/q/ui/}{WIRR}
  \end{list2}
\end{list0}

\slide{VOTable}

\begin{SaveVerbatim}{votable}
<?xml version='1.0'?>
<VOTABLE version="1.4" xmlns="http://www.ivoa.net/xml/VOTable/v1.3">
  <RESOURCE>
    <TABLE name="dr3lite" nrows="10">
      <DESCRIPTION>This data is from Gaia DR3</DESCRIPTION>
      <FIELD name="source_id" datatype="long" ucd="meta.id;meta.main">
        <DESCRIPTION>Gaia DR3 unique source identifier.</DESCRIPTION>
      </FIELD>
      <FIELD name="ra" datatype="float" ucd="pos.eq.ra;meta.main" unit="deg">
        <DESCRIPTION>Barycentric Right Ascension in ICRS at epoch J2016.0</DESCRIPTION>
      </FIELD>
      <FIELD name="dec" datatype="float" ucd="pos.eq.dec;meta.main" unit="deg">
        <DESCRIPTION>Barycentric Declination in ICRS at epoch J2016.0</DESCRIPTION>
      </FIELD>
      <FIELD name="parallax" datatype="float" ucd="pos.parallax" unit="mas">
        <DESCRIPTION>Absolute barycentric stellar parallax of the source at J2016.0.</DESCRIPTION>
      </FIELD>
      <DATA>
        <TABLEDATA>
          <TR><TD>4267180339403392768</TD> <TD>286.71692</TD> <TD>0.2761946</TD>  <TD>1.0849239</TD></TR>
          <TR><TD>5252403815119316480</TD> <TD>152.62506</TD> <TD>-64.2677</TD>   <TD>0.8543564</TD></TR>
          <TR><TD>1937745177867542656</TD> <TD>348.43497</TD> <TD>43.940704</TD>  <TD>1.0420077</TD></TR>
          <TR><TD>5971301282285218048</TD> <TD>252.72972</TD> <TD>-37.917213</TD> <TD>0.58765984</TD></TR>
          <TR><TD>5953201637243132416</TD> <TD>260.6788</TD>  <TD>-44.788773</TD> <TD>-0.2933756</TD></TR>
          <TR><TD>6013771224556727040</TD> <TD>231.36102</TD> <TD>-35.700317</TD> <TD/></TR>
          <TR><TD>6125916630291830144</TD> <TD>186.18074</TD> <TD>-50.75659</TD>  <TD>0.43694037</TD></TR>
          <TR><TD>6018228919573319936</TD> <TD>245.57784</TD> <TD>-39.0594</TD>   <TD>2.3795636</TD></TR>
          <TR><TD>5203483274312495872</TD> <TD>144.84912</TD> <TD>-77.016785</TD> <TD>0.35218337</TD></TR>
          <TR><TD>4162615512253865856</TD> <TD>263.4329</TD>  <TD>-12.341759</TD> <TD>-0.2832363</TD></TR>
        </TABLEDATA>
      </DATA>
    </TABLE>
  </RESOURCE>
</VOTABLE>
\end{SaveVerbatim}

\begin{picture}(30,0)
  \put(17,-16){{\color{brown}\tiny\scalebox{0.8}{\BUseVerbatim{votable}}}}
\end{picture}
\vspace*{-1.5cm}

\begin{list0}
  \item VOTable is a transport format for tabular data in the VO
  \begin{list2}
    \item Used for moving lots of data around in the VO
    \item XML-based
    \item Suitable for streaming
    \item Can carry VO/astronomy-related metadata
    \item Can be read by Astropy, TOPCAT etc
    \item Mostly {\bf you don't need to understand the details}
  \end{list2}
\end{list0}

\slide{Simple positional queries}

\begin{picture}(30,0)
  \put(0,-17){\includegraphics[height=8cm]{m31-cone.pdf}}
  \put(9,-17){\includegraphics[height=8cm]{cone-table.png}}
  \put(24,-17){\includegraphics[height=16cm]{cone-window.png}}
\end{picture}
\vspace*{-1.5cm}

\begin{list0}
  \item \textbf{Cone} Search,
        \textbf{S}imple \textbf{I}mage \textbf{A}ccess,
        \textbf{S}imple \textbf{S}pectral \textbf{A}ccess
  \begin{list2}
    \item Give the service a sky position (RA, Dec) and some kind of radius \\
          (details differ a bit between Cone, SIA, SSA)
    \item The service gives you back a (VO)Table with a row for each result
    \begin{description}
      \item[Cone:] list of catalogue entries (e.g.\ sources) near that position
      \item[SIA:]  list of images near that position
                   (one column has a URL for the image file)
      \item[SSA:]  list of spectra near that position
                   (one column has a URL for the spectrum file)
    \end{description}
  \end{list2}
\end{list0}

\slide{TAP/ADQL}

\begin{picture}(30,0)
  \put(20,-16){\includegraphics[height=16cm]{TapTableLoadDialog_query.png}}
  \put(20,-17){{\footnotesize\sl See also my dev group talk from 3 May 2024:
                \bhref{https://github.com/astro-group-bristol/developer-group/files/15199523/sql.pdf}
                      {SQL (and ADQL and TAP)}}}
\end{picture}
\vspace*{-1.5cm}

\begin{list0}
  \item TAP: Table Access Protocol
  \begin{list2}
    \item Execute an SQL-like query on a remote database
    \item Provide information to the user (to understand how to query)
    \begin{list3}
      \item Table metadata, ADQL extra functions, example queries, ...
    \end{list3}
  \end{list2}
  \item ADQL: Astronomical Data Query Language
  \begin{list2}
    \item SQL-like language you write queries in
    \item Mostly just a standardised dialect of SQL, \\
          a few astro-specific features (spherical geometry)
  \end{list2}
  \item Not so simple but very powerful
  \begin{list2}
    \item You can do very flexible and powerful things
    \item Simple queries, selections, crossmatches, combine multiple tables, \\
          local uploads, selections, histogram/sky map calculations, ...
  \end{list2}
  \item Clients
  \begin{list2}
    \item I recommend TOPCAT!
    \item Also
          \bhref{https://pyvo.readthedocs.io/en/latest/dal/index.html\#pyvo-tap}
                {PyVO},
          service-specific web pages etc
  \end{list2}
\end{list0}

\slide{HiPS}

\begin{picture}(30,0)
  \put(22,-17){\includegraphics[width=15cm]{aladin-lite.png}}
  \put(8,-17.5){\includegraphics[height=4cm]{HEALPix.png}}
\end{picture}
\vspace*{-1.5cm}

\begin{list0}
  \item All-sky hierarchical imagery
        (\textbf{Hi}erarchical \textbf{P}rogressive \textbf{S}urvey)
  \begin{list2big}
    \item Like Google Earth, but looking up (usually)
    \item Based on HEALPix tiling (doesn't blow up at the poles)
    \item HiPS datasets exist for
          lots of sky surveys image archives
    \begin{list3}
      \item Also quite a few non-sky image sets too (Mars, the Moon, ...)
    \end{list3}
    \item Usually access it using a dedicated all-sky viewer, \\
          mostly Aladin variants:
    \begin{list3}
      \item \bhref{https://aladin.cds.unistra.fr/}{Aladin}:
            desktop java tool
      \item \bhref{https://aladin.cds.unistra.fr/AladinLite/doc/}{Aladin Lite}:
            web-based (super easy to embed in your own web pages)
      \item \bhref{https://github.com/cds-astro/ipyaladin}{ipyaladin}:
            stick it in a Jupyter notebook
      \item \bhref{https://sky.esa.int/}{ESASky}:
            AladinLite-based interface to ESA-curated sky survey data
    \end{list3}
    \item You can also do cutouts: 
          \bhref{https://alasky.cds.unistra.fr/hips-image-services/hips2fits}
                {HiPS2FITS}
    \item HiPS exist for some spectral cube datasets too
  \end{list2big}
\end{list0}

\slide{SAMP}

\begin{picture}(30,0)
  \put(18,-17){\includegraphics[height=7cm]{linked-ir.png}}
\end{picture}
\vspace*{-1.5cm}

\begin{list0}
  \item Simple Application Messaging Protocol
  \begin{list2big}
    \item Lets astro applications talk to each other
    \item Participating clients:
    \begin{list3}
      \item TOPCAT, Aladin, Astropy, ds9, Aladin-Lite, CASSIS, SPLAT, \\
            various archive web pages, your python script or web page, ...
    \end{list3}
    \item Protocol design: \includegraphics[height=3ex]{hub.png}
    \begin{list3}
      \item Publish/subscribe message passing,
            Remote Procedure Calls using XML-RPC
      \item SAMP ``Hub'' sits on the desktop and mediates communications
      \item When it works well it's quite neat \Smiley
      \item Sometimes it doesn't work
            (network setup, OS issues, IPC funnies, ...) \Frowny
    \end{list3}
    \item Types of messages sent (MTypes):
    \begin{list3}
      \item Mostly quite generic
      \item e.g.: ``load this table'', ``point at this sky position'', \\
                  ``highlight row \#N of the table I sent you''
    \end{list3}
  \end{list2big}
\end{list0}

\slide{MOC}

\begin{picture}(30,0)
  \put(18,-7){\includegraphics[height=5.2cm]{HEALPix.png}}
  \put(16,-17){\includegraphics[height=10cm]{moc.png}}
  \put(1,-17){\includegraphics[height=8cm]{ngcs.pdf}}
\end{picture}
\vspace*{-1.5cm}

\begin{list0}
  \item Multi-Order Coverage maps
  \begin{list2big}
    \item Format for describing sky coverage
    \item Based on HEALPix tiling
    \item Serialize as FITS or ASCII
    \item Memory-efficient: use low-order (big) pixels \\
          where possible, high-order (small) pixels for detail
    \item Very fast calculations (search, intersection, union)
    \item Various libraries,
          e.g.\ \bhref{https://cds-astro.github.io/mocpy/}{MOCPy}
  \end{list2big}
\end{list0}

\slide{UCD/VOUnits}

\begin{list0}
  \item Column metadata so that you and your computers can understand tables
  \begin{list2}
    \item Not really necessary for the 3-column CSV file you just wrote
    \item Quite useful for a 1000-column monster from an archive you
          never heard of
  \end{list2}
  \item[] \textbf{UCD:} Uniform Content Descriptors
  \begin{list2}
    \item Machine-readable semantic tags for columns
    \item Hierarchical, can be stuck together with semicolons
    \begin{list3}
      \item \texttt{pos.eq.ra}, \texttt{pos.eq.dec},
            \texttt{meta.id;meta.main},
            \texttt{em.opt.V},
            \texttt{obs.atmos.turbulence.isoplanatic}, ...
    \end{list3}
    \item There's a big list at
          \bhref{https://www.ivoa.net/documents/UCD1+/}{EN-UCDlist-1.6}
  \end{list2}
  \item[] \textbf{VOUnits:} Units for the VO
  \begin{list2}
    \item Just a standard way to encode units understood by VO tools
    \begin{list3}
      \item \texttt{m/s**2}, \texttt{m.s**-2}, \texttt{mJy}, \texttt{log(GHz)},
            ...
    \end{list3}
    \item Mostly sensible but a few surprises:
          no whitespace, ``.'' required for multiplication,
          ``\texttt{**}'' for powers
    \item Specification at
          \bhref{https://www.ivoa.net/documents/VOUnits/}{REC-VOUnits-1.1}
  \end{list2}
\end{list0}

% \slide{ObsCore}

\slide{And there's more!}

\begin{list0}
  \item Some other VO standards:
  \begin{list2}
    \item DataLink
    \item DALI
    \item SLAP
    \item VOEvent
    \item SODA
    \item ObjObsSAP
    \item EPN-TAP
    \item LineTAP
%   \item STC
%   \item VODML
%   \item MIVOT
%   \item UWS
%   \item SSO
%   \item GMS
%   \item VOSI
%   \item[] ...
    \item[] ...
  \end{list2}
  \item Plus non-standard but very useful VO-friendly services:
  \begin{list2}
    \item VizieR
    \item CDS X-Match
    \item VOSA
    \item SIMBAD
    \item[] ...
  \end{list2}
\end{list0}


\newpage

\begin{center}
  \includegraphics[width=35cm]{ivoa_logoc.jpg}
\end{center}

\label{lastPage}
   
\ifrubric

\newcommand{\aobSlide}[1]{
\newpage
\rightfooter{}
\MyLogo{} 
\begin{picture}(30,0)
  #1
\end{picture} 
\bigword{\ \ AOB?}
}
\aobSlide{}


\fi

\end{document}
